%	Copyright 2012 - 2015 Srinivas Kowtal (srinivaskay@gmail.com)
%
% Original copyright information follows below
%
%% start of file `template.tex'.
%% Copyright 2006-2010 Xavier Danaux (xdanaux@gmail.com).
%
% This work may be distributed and/or modified under the
% conditions of the LaTeX Project Public License version 1.3c,
% available at http://www.latex-project.org/lppl/.

\documentclass[11pt,letterpaper]{moderncv}

%\usepackage{hyperref}
\usepackage{color}
\usepackage{fontspec}

% moderncv themes
% optional argument are 'blue' (default), 'orange', 'red', 'green', 'grey' and 'roman' (for roman fonts, instead of sans serif fonts)
%\moderncvtheme[blue]{casual}                
\moderncvtheme[blue]{classic}
%\moderncvtheme[blue]{casual}                

% DOCUMENT LAYOUT
\usepackage[scale=0.85]{geometry}

\setlength{\hintscolumnwidth}{2.3cm} % if you want to change the width of the column with the dates
%\AtBeginDocument{\setlength{\maketitlenamewidth}{6cm}}  % only for the classic theme, if you want to change the width of your name placeholder (to leave more space for your address details
\AtBeginDocument{\recomputelengths} % required when changes are made to page layout lengths

% CHARACTER ENCODING
\usepackage{xunicode}
\usepackage{xltxtra}
\usepackage[utf8]{inputenc}                 % replace by the encoding you are using
\defaultfontfeatures{Mapping=tex-text}      % converts LaTeX specials (``quotes'' --- dashes
                                            % etc.) to unicode
% FONTS
%\setromanfont [Ligatures={Common}, Numbers={OldStyle}]{Linux Libertine O}
%\setmonofont[Scale=0.8]{DejaVuSansMono}
%\setsansfont[Scale=0.9]{OptimaLTStd}

% ---- CUSTOM AMPERSAND
%\newcommand{\amper}{{\fontspec[Scale=.95]{Adobe Caslon Pro}\selectfont\itshape\&}}

% ---- MARGIN YEARS
%\newcommand{\years}[1]{\marginpar{\scriptsize #1}}

% personal data
\firstname{André}
\familyname{Abadesso}
%\title{Srini Kowtal's Resume}                     % optional, remove the line if not wanted
\address{Rua dos Jacarandás 300}{Barra da Tijuca, 22776-050} % optional, remove the line if not wanted
\mobile{\texttt{(21)99985-8806}}              % optional, remove the line if not wanted
%\phone{\texttt{+91-9000012854}}               % optional, remove the line if not wanted
%\fax{fax (optional)}                          % optional, remove the line if not wanted
\email{andre@lab21k.com.br}                      % optional, remove the line if not wanted
%\extrainfo{https://github.com/zhaphod} % optional, remove the line if not wanted

\homepage{github.com/andreabadesso}                % optional, remove the line if not wanted
%\extrainfo{Programmer} % optional, remove the line if not wanted
%\photo[64pt]{dont_panic.png}% '64pt' is the height the picture must be resized to and 'picture' is the name of the picture file; optional, remove the line if not wanted
%\quote{ Circuit bender, kernel hacker, robot builder, urban explorer, trombone playerer.}
% to show numerical labels in the bibliography; only useful if you make citations in your resume
%\makeatletter
%\renewcommand*{\bibliographyitemlabel}{\@biblabel{\arabic{enumiv}}}
%\makeatother

% bibliography with mutiple entries
%\usepackage{multibib}
%\newcites{book,misc}{{Books},{Others}}

\nopagenumbers{}% uncomment to suppress automatic page numbering for CVs longer than one page
%----------------------------------------------------------------------------------
%            content
%----------------------------------------------------------------------------------
\begin{document}

\maketitle

\section{Qualificações}
\cvline{}
	{
		\begin{itemize}
		    \item Contribuidor em diversos projetos open-source
		    \item Extenso conhecimento em padrões de design para desenvolvimento web
		    \item Desenvolvedor Full-Stack, com grande experiência em todas as camadas de uma aplicação web
		\end{itemize}}{}{}{}{}

\section{Experiência Técnica}
	 \cvline{Linguagens}{Javascript, Python, Java, Scala}
	 \cvline{OS}{OSX, Linux, FreeBSD}
	 \cvline{Bancos de Dados}{PostgreSQL, MSSQL, MySQL, MongoDB, REDIS, Neo4j}
	 \cvline{Ferramentas}{Vim, Tmux, Zsh, Grunt, Bower, Pip}
	 \cvline{Controle de versão}{Git, SVN}
	 \cvline{Frameworks}{AngularJS, NodeJS, ReactJS, Django, Flask, Tornado, Play (Java e Scala)}
	 \cvline{HW}{Raspberry Pi, Arduino, PIC (Turbo-C)}
	 
\cvline  {}{}

\section{Educação}
	 \cventry{}{Bacharelado em Sistemas de Informação}{PUC-RIO}{}{}{}
	 \cventry{}{Cultura Inglesa}{Curso de inglês pela Cultura Inglesa}{}{}{}

\cvline  {}{}

\section{Linguagens}
\cvlanguage{Português}{Nativo}{}
\cvlanguage{Inglês}{Fluente}{}
\cvlanguage{Espanhol}{Básico}{}


\clearpage

\section{Experiência Profissional} 
		\cventry {Jan14-Atual}{21k}{}{}{}{}
    \subsection{Analista Sr. / Gerente de Produto}	
    \cvlistitem  {Coordenei o desenvolvimento do aplicativo e plataforma iCongresso de gestão de congressos (http://icongresso.com/). O sistema auxilia organizadores de eventos e congressos na coordenação do evento e ao mesmo tempo serve como fonte principal de informação para participantes dele. A camada de negócios foi desenvolvida em Python, utilizando a framework Django. A camada de apresentação foi desenvolvida utilizando Java e Objective-C. A camada de dados utiliza MongoDB. Foi utilizado armazenamento em núvem S3 da Amazon (AWS) para armazenar os artigos e imagens dos palestrantes.}		
		\cvlistitem  {Desenvolvi a plataforma IndiQ (http://indiq.me/) - Comunidade online que busca fazer a ponte entre lojas, produtos e pessoas em uma única plataforma social, onde os usuários podem ter uma experiência personalizada de descoberta e compartilhamento de produtos das lojas de ecommerce no Brasil. Trabalhei diretamente no desenvolvimento full-stack da aplicação, que utiliza Python (Django) na camada de negócios e AngularJS na camada de apresentação.}

		\cvlistitem  {Liderei uma equipe de 3 desenvolvedores e trabalhei diretamente no desenvolvimento da aplicação S4C, sistema georeferenciado de gestão situacional.}

	\cventry{Nov13-Out14}{Centro de Operações Rio}{}{}{}{}
    \subsection{Desenvolvedor}
		\cvlistitem  {Implementei a integração do sistema Geoportal com os serviços de análise de trânsito do Waze. Integração ponto a ponto utilizando um script desenvolvido em Python e executado através de um Scheduler também desenvolvido por mim.}
    \cvlistitem  {Idealizei e implementei em forma de extensão ao sistema Geoportal um mosaico interativo de cameras, que permite ao operador visualizar diversas cameras da cidade em uma interface única.}
    \cvlistitem  {Idealizei e implementei uma plataforma de criação e monitoramento de rotas importantes na cidade do Rio de Janeiro para facilitar a operação do Centro de Operações utilizando a API do Google Maps for Business em Python, em cima do framework Django utilizando PostgreSQL na camada de dados. Na camada de apresentação, foi utilizada a framework AngularJS e o mapa do Google Maps. Este sistema foi extensamente utilizado durante a Copa do Mundo de 2014 para monitorar as rotas dos deslocamentos que as seleções fizeram dos hotéis até os campos de treinamento/partida.}
    \cvlistitem{Integrei a plataforma de criação e monitoramento de Rotas com os relógios digitais da cidade do Rio de Janeiro. Cada relógio digital exibe o tempo de duas possíveis rotas partindo de sua posição geográfica. O objetivo é auxiliar o motorista a escolher a rota com o menor tempo e com isso, melhorar o trânsito na cidade.}
    \cvlistitem{Trabalhei no desenvolvimento de um aplicativo para o Google Glass que integra funcionalidades do Geoportal como Datamining, Tarefas e Envio de Incidentes. Um MVP foi construido onde era possível enviar fotos direto para o sistema e atualizar a posição do usuário em tempo real.}
    \cvlistitem{Idealizei e desenvolvi aplicativo mobile para as plataformas iOS e Android de uso interno do Centro de Operações para notificação de problemas na cidade por parte de suas bases de apoio espalhadas pela cidade.}

\clearpage

 \cventry {Mar13-Set13}{GEO-RIO}{}{}{}{}
    \subsection{Estagiário}
    \cvlistitem  {Atuei no desenvolvimento e manutenção do programa GEORISQ2, sistema GIS que cuida do georeferenciamento de obras da fundação e mapeamento das áreas de risco da cidade. Como a empresa não possuia uma equipe de desenvolvimento, fui o único desenvolvedor no periodo em contato com a plataforma.}
    \cvlistitem  {Desenvolvi, em parceiria com a empresa Squitter Ambiental, aplicativo para Android e Iphone contendo todas as funcionalidades do site Alerta-Rio (http://www0.rio.rj.gov.br/alertario/). Sistema desenvolvido em AngularJS utilizando uma framework de webapps (Ionic).}
    
    \cventry {Jan13-Mai13}{Ninho dos Corujas - http://www.ninhodoscorujas.com.br}{}{}{}{}
    \subsection{\small\emph{Freelance, Desenvolvedor Fullstack}}
    \cvlistitem  {Desenvolvimento fullstack do ecommerce Ninho dos Corujas}
    \cvlistitem  {Desenvolvi o backend e o frontend do website Ninho dos Corujas. O Backend da aplicação é uma API RESTful que utiliza a framework Django, em Python. A camada de apresentação da aplicação utiliza AngularJS como framework principal, que cuida do carrinho do usuário e de toda a lógica do chá de bebê online.}
    \cventry {Fev13-Mar13}{CornetaRIO}{}{}{}{}
    \subsection{\small\emph{Desenvolvedor Full-Stack}}
    \cvlistitem{Desenvolvimento, em parceiria com a empresa Synapse, Inovadora da incubadora Genesis da PUC-RIO do aplicativo mobile CornetaRIO para as plataformas iOS e Android.  É um instagram para alertas de mobilidade urbana, através dele o usuário pode compartilhar os problemas da cidade e comentar em outras publicações com sugestões sobre o que melhorar. O aplicativo foi desenvolvido em AngularJS na camada de apresentação, NodeJS na camada de negócios e MongoDB na camada de dados.}
    

\cvline  {}{}

\section{Cursos}
    \cventry{2015}{Full Stack Web Developer Nanodegree}{Udacity}{}{}{}
    \cventry{2015}{Business Process Application Development}{Bonitasoft}{}{}{}
    \cventry{2014}{Stanford - Algorithms: Design and Analysis, Part 1}{Coursera}{}{}{}
	\cventry{2014}{Stanford - Algorithms: Design and Analysis, Part 2}{Coursera}{}{}{}
	\cventry{2014}{Advanced Node.js Development}{Udemy}{}{}{}	
    \cventry{2013}{Stanford - Startup Engineering}{Coursera}{}{}{}

\cvline  {}{}
\section{Certificados}
	%\cventry {At QCom}{I was given ``QualStar'' award for contributions to the LGE V510 Tablet project}{}{}{}{}

	\subsection{\emph{Cambridge English: First (FCE)}}
	\subsection{\emph{Bonitasoft: Business Process Application Development}}
\cvline  {}{}

\cvline {}{}

\begin{center}
\color{gray}\Large{$\nu\epsilon$ $\pi\alpha\beta o \rho$}
\end{center}
	
\end{document}


%\section{Experience}
%\subsection{Vocational}
%\cventry{year--year}{Job title}{Employer}{City}{}{General description no longer than 1--2 lines.\newline%
%Detailed achievements:%
%\begin{itemize}%
%\item Achievement 1;
%\item Achievement 2, with sub-achievements:
%  \begin{itemize}%
%  \item Sub-achievement (a);
%  \item Sub-achievement (b), with sub-sub-achievements (don't do this!);
%    \begin{itemize}
%    \item Sub-sub-achievement i;
%    \item Sub-sub-achievement ii;
%    \item Sub-sub-achievement iii;
%    \end{itemize}
%  \item Sub-achievement (c);
%  \end{itemize}
%\item Achievement 3.
%\end{itemize}}
%\cventry{year--year}{Job title}{Employer}{City}{}{Description line 1\newlineDescription line 2}
%\subsection{Miscellaneous}
%\cventry{year--year}{Job title}{Employer}{City}{}{Description}


%\section{Computer skills}
%\cvcomputer{category 1}{XXX, YYY, ZZZ}{category 4}{XXX, YYY, ZZZ}
%\cvcomputer{category 2}{XXX, YYY, ZZZ}{category 5}{XXX, YYY, ZZZ}
%\cvcomputer{category 3}{XXX, YYY, ZZZ}{category 6}{XXX, YYY, ZZZ}

%\section{Interests}
%\cvline{hobby 1}{\small Description}
%\cvline{hobby 2}{\small Description}
%\cvline{hobby 3}{\small Description}

%\section{Extra 1}
%\cvlistitem{Item 1}
%\cvlistitem{Item 2}
%\cvlistitem[+]{Item 3}            % optional other symbol

%\renewcommand{\listitemsymbol}{-} % change the symbol for lists

%\section{Extra 2}
%\cvlistdoubleitem{Item 1}{Item 4}
%\cvlistdoubleitem{Item 2}{Item 5 \cite{book1}}
%\cvlistdoubleitem{Item 3}{}

% Publications from a BibTeX file without multibib\renewcommand*{\bibliographyitemlabel}{\@biblabel{\arabic{enumiv}}}% for BibTeX numerical labels
%\nocite{*}
%\bibliographystyle{plain}
%\bibliography{publications}       % 'publications' is the name of a BibTeX file

% Publications from a BibTeX file using the multibib package
%\section{Publications}
%\nocitebook{book1,book2}
%\bibliographystylebook{plain}
%\bibliographybook{publications}   % 'publications' is the name of a BibTeX file
%\nocitemisc{misc1,misc2,misc3}
%\bibliographystylemisc{plain}
%\bibliographymisc{publications}   % 'publications' is the name of a BibTeX file


%% end of file `template_en.tex'.

